% Options for packages loaded elsewhere
\PassOptionsToPackage{unicode}{hyperref}
\PassOptionsToPackage{hyphens}{url}
\PassOptionsToPackage{dvipsnames,svgnames,x11names}{xcolor}
%
\documentclass[
  letterpaper,
  DIV=11,
  numbers=noendperiod]{scrreprt}

\usepackage{amsmath,amssymb}
\usepackage{iftex}
\ifPDFTeX
  \usepackage[T1]{fontenc}
  \usepackage[utf8]{inputenc}
  \usepackage{textcomp} % provide euro and other symbols
\else % if luatex or xetex
  \usepackage{unicode-math}
  \defaultfontfeatures{Scale=MatchLowercase}
  \defaultfontfeatures[\rmfamily]{Ligatures=TeX,Scale=1}
\fi
\usepackage{lmodern}
\ifPDFTeX\else  
    % xetex/luatex font selection
\fi
% Use upquote if available, for straight quotes in verbatim environments
\IfFileExists{upquote.sty}{\usepackage{upquote}}{}
\IfFileExists{microtype.sty}{% use microtype if available
  \usepackage[]{microtype}
  \UseMicrotypeSet[protrusion]{basicmath} % disable protrusion for tt fonts
}{}
\makeatletter
\@ifundefined{KOMAClassName}{% if non-KOMA class
  \IfFileExists{parskip.sty}{%
    \usepackage{parskip}
  }{% else
    \setlength{\parindent}{0pt}
    \setlength{\parskip}{6pt plus 2pt minus 1pt}}
}{% if KOMA class
  \KOMAoptions{parskip=half}}
\makeatother
\usepackage{xcolor}
\setlength{\emergencystretch}{3em} % prevent overfull lines
\setcounter{secnumdepth}{3}
% Make \paragraph and \subparagraph free-standing
\ifx\paragraph\undefined\else
  \let\oldparagraph\paragraph
  \renewcommand{\paragraph}[1]{\oldparagraph{#1}\mbox{}}
\fi
\ifx\subparagraph\undefined\else
  \let\oldsubparagraph\subparagraph
  \renewcommand{\subparagraph}[1]{\oldsubparagraph{#1}\mbox{}}
\fi

\usepackage{color}
\usepackage{fancyvrb}
\newcommand{\VerbBar}{|}
\newcommand{\VERB}{\Verb[commandchars=\\\{\}]}
\DefineVerbatimEnvironment{Highlighting}{Verbatim}{commandchars=\\\{\}}
% Add ',fontsize=\small' for more characters per line
\usepackage{framed}
\definecolor{shadecolor}{RGB}{248,248,248}
\newenvironment{Shaded}{\begin{snugshade}}{\end{snugshade}}
\newcommand{\AlertTok}[1]{\textcolor[rgb]{0.94,0.16,0.16}{#1}}
\newcommand{\AnnotationTok}[1]{\textcolor[rgb]{0.56,0.35,0.01}{\textbf{\textit{#1}}}}
\newcommand{\AttributeTok}[1]{\textcolor[rgb]{0.13,0.29,0.53}{#1}}
\newcommand{\BaseNTok}[1]{\textcolor[rgb]{0.00,0.00,0.81}{#1}}
\newcommand{\BuiltInTok}[1]{#1}
\newcommand{\CharTok}[1]{\textcolor[rgb]{0.31,0.60,0.02}{#1}}
\newcommand{\CommentTok}[1]{\textcolor[rgb]{0.56,0.35,0.01}{\textit{#1}}}
\newcommand{\CommentVarTok}[1]{\textcolor[rgb]{0.56,0.35,0.01}{\textbf{\textit{#1}}}}
\newcommand{\ConstantTok}[1]{\textcolor[rgb]{0.56,0.35,0.01}{#1}}
\newcommand{\ControlFlowTok}[1]{\textcolor[rgb]{0.13,0.29,0.53}{\textbf{#1}}}
\newcommand{\DataTypeTok}[1]{\textcolor[rgb]{0.13,0.29,0.53}{#1}}
\newcommand{\DecValTok}[1]{\textcolor[rgb]{0.00,0.00,0.81}{#1}}
\newcommand{\DocumentationTok}[1]{\textcolor[rgb]{0.56,0.35,0.01}{\textbf{\textit{#1}}}}
\newcommand{\ErrorTok}[1]{\textcolor[rgb]{0.64,0.00,0.00}{\textbf{#1}}}
\newcommand{\ExtensionTok}[1]{#1}
\newcommand{\FloatTok}[1]{\textcolor[rgb]{0.00,0.00,0.81}{#1}}
\newcommand{\FunctionTok}[1]{\textcolor[rgb]{0.13,0.29,0.53}{\textbf{#1}}}
\newcommand{\ImportTok}[1]{#1}
\newcommand{\InformationTok}[1]{\textcolor[rgb]{0.56,0.35,0.01}{\textbf{\textit{#1}}}}
\newcommand{\KeywordTok}[1]{\textcolor[rgb]{0.13,0.29,0.53}{\textbf{#1}}}
\newcommand{\NormalTok}[1]{#1}
\newcommand{\OperatorTok}[1]{\textcolor[rgb]{0.81,0.36,0.00}{\textbf{#1}}}
\newcommand{\OtherTok}[1]{\textcolor[rgb]{0.56,0.35,0.01}{#1}}
\newcommand{\PreprocessorTok}[1]{\textcolor[rgb]{0.56,0.35,0.01}{\textit{#1}}}
\newcommand{\RegionMarkerTok}[1]{#1}
\newcommand{\SpecialCharTok}[1]{\textcolor[rgb]{0.81,0.36,0.00}{\textbf{#1}}}
\newcommand{\SpecialStringTok}[1]{\textcolor[rgb]{0.31,0.60,0.02}{#1}}
\newcommand{\StringTok}[1]{\textcolor[rgb]{0.31,0.60,0.02}{#1}}
\newcommand{\VariableTok}[1]{\textcolor[rgb]{0.00,0.00,0.00}{#1}}
\newcommand{\VerbatimStringTok}[1]{\textcolor[rgb]{0.31,0.60,0.02}{#1}}
\newcommand{\WarningTok}[1]{\textcolor[rgb]{0.56,0.35,0.01}{\textbf{\textit{#1}}}}

\providecommand{\tightlist}{%
  \setlength{\itemsep}{0pt}\setlength{\parskip}{0pt}}\usepackage{longtable,booktabs,array}
\usepackage{calc} % for calculating minipage widths
% Correct order of tables after \paragraph or \subparagraph
\usepackage{etoolbox}
\makeatletter
\patchcmd\longtable{\par}{\if@noskipsec\mbox{}\fi\par}{}{}
\makeatother
% Allow footnotes in longtable head/foot
\IfFileExists{footnotehyper.sty}{\usepackage{footnotehyper}}{\usepackage{footnote}}
\makesavenoteenv{longtable}
\usepackage{graphicx}
\makeatletter
\def\maxwidth{\ifdim\Gin@nat@width>\linewidth\linewidth\else\Gin@nat@width\fi}
\def\maxheight{\ifdim\Gin@nat@height>\textheight\textheight\else\Gin@nat@height\fi}
\makeatother
% Scale images if necessary, so that they will not overflow the page
% margins by default, and it is still possible to overwrite the defaults
% using explicit options in \includegraphics[width, height, ...]{}
\setkeys{Gin}{width=\maxwidth,height=\maxheight,keepaspectratio}
% Set default figure placement to htbp
\makeatletter
\def\fps@figure{htbp}
\makeatother
% definitions for citeproc citations
\NewDocumentCommand\citeproctext{}{}
\NewDocumentCommand\citeproc{mm}{%
  \begingroup\def\citeproctext{#2}\cite{#1}\endgroup}
\makeatletter
 % allow citations to break across lines
 \let\@cite@ofmt\@firstofone
 % avoid brackets around text for \cite:
 \def\@biblabel#1{}
 \def\@cite#1#2{{#1\if@tempswa , #2\fi}}
\makeatother
\newlength{\cslhangindent}
\setlength{\cslhangindent}{1.5em}
\newlength{\csllabelwidth}
\setlength{\csllabelwidth}{3em}
\newenvironment{CSLReferences}[2] % #1 hanging-indent, #2 entry-spacing
 {\begin{list}{}{%
  \setlength{\itemindent}{0pt}
  \setlength{\leftmargin}{0pt}
  \setlength{\parsep}{0pt}
  % turn on hanging indent if param 1 is 1
  \ifodd #1
   \setlength{\leftmargin}{\cslhangindent}
   \setlength{\itemindent}{-1\cslhangindent}
  \fi
  % set entry spacing
  \setlength{\itemsep}{#2\baselineskip}}}
 {\end{list}}
\usepackage{calc}
\newcommand{\CSLBlock}[1]{\hfill\break\parbox[t]{\linewidth}{\strut\ignorespaces#1\strut}}
\newcommand{\CSLLeftMargin}[1]{\parbox[t]{\csllabelwidth}{\strut#1\strut}}
\newcommand{\CSLRightInline}[1]{\parbox[t]{\linewidth - \csllabelwidth}{\strut#1\strut}}
\newcommand{\CSLIndent}[1]{\hspace{\cslhangindent}#1}

% Soul package to handle highlighting (see hl.py3 filter)
\usepackage{soul}

% For tables generated by the gt package
\usepackage{colortbl}
\KOMAoption{captions}{tableheading}
\makeatletter
\@ifpackageloaded{tcolorbox}{}{\usepackage[skins,breakable]{tcolorbox}}
\@ifpackageloaded{fontawesome5}{}{\usepackage{fontawesome5}}
\definecolor{quarto-callout-color}{HTML}{909090}
\definecolor{quarto-callout-note-color}{HTML}{0758E5}
\definecolor{quarto-callout-important-color}{HTML}{CC1914}
\definecolor{quarto-callout-warning-color}{HTML}{EB9113}
\definecolor{quarto-callout-tip-color}{HTML}{00A047}
\definecolor{quarto-callout-caution-color}{HTML}{FC5300}
\definecolor{quarto-callout-color-frame}{HTML}{acacac}
\definecolor{quarto-callout-note-color-frame}{HTML}{4582ec}
\definecolor{quarto-callout-important-color-frame}{HTML}{d9534f}
\definecolor{quarto-callout-warning-color-frame}{HTML}{f0ad4e}
\definecolor{quarto-callout-tip-color-frame}{HTML}{02b875}
\definecolor{quarto-callout-caution-color-frame}{HTML}{fd7e14}
\makeatother
\makeatletter
\@ifpackageloaded{bookmark}{}{\usepackage{bookmark}}
\makeatother
\makeatletter
\@ifpackageloaded{caption}{}{\usepackage{caption}}
\AtBeginDocument{%
\ifdefined\contentsname
  \renewcommand*\contentsname{Índice}
\else
  \newcommand\contentsname{Índice}
\fi
\ifdefined\listfigurename
  \renewcommand*\listfigurename{Lista de Figuras}
\else
  \newcommand\listfigurename{Lista de Figuras}
\fi
\ifdefined\listtablename
  \renewcommand*\listtablename{Lista de Tabelas}
\else
  \newcommand\listtablename{Lista de Tabelas}
\fi
\ifdefined\figurename
  \renewcommand*\figurename{Figura}
\else
  \newcommand\figurename{Figura}
\fi
\ifdefined\tablename
  \renewcommand*\tablename{Tabela}
\else
  \newcommand\tablename{Tabela}
\fi
}
\@ifpackageloaded{float}{}{\usepackage{float}}
\floatstyle{ruled}
\@ifundefined{c@chapter}{\newfloat{codelisting}{h}{lop}}{\newfloat{codelisting}{h}{lop}[chapter]}
\floatname{codelisting}{Listagem}
\newcommand*\listoflistings{\listof{codelisting}{Lista de Listagens}}
\makeatother
\makeatletter
\makeatother
\makeatletter
\@ifpackageloaded{caption}{}{\usepackage{caption}}
\@ifpackageloaded{subcaption}{}{\usepackage{subcaption}}
\makeatother
\ifLuaTeX
\usepackage[bidi=basic]{babel}
\else
\usepackage[bidi=default]{babel}
\fi
\babelprovide[main,import]{portuguese}
% get rid of language-specific shorthands (see #6817):
\let\LanguageShortHands\languageshorthands
\def\languageshorthands#1{}
\ifLuaTeX
  \usepackage{selnolig}  % disable illegal ligatures
\fi
\usepackage{bookmark}

\IfFileExists{xurl.sty}{\usepackage{xurl}}{} % add URL line breaks if available
\urlstyle{same} % disable monospaced font for URLs
\hypersetup{
  pdftitle={Elementos de Matemática Discreta para Computação: Soluções},
  pdfauthor={Anamaria Gomide, Jorge Stolfi, Fernando Náufel},
  pdflang={pt},
  colorlinks=true,
  linkcolor={blue},
  filecolor={Maroon},
  citecolor={Blue},
  urlcolor={Blue},
  pdfcreator={LaTeX via pandoc}}

\title{Elementos de Matemática Discreta para Computação: Soluções}
\author{Anamaria Gomide, Jorge Stolfi, Fernando Náufel}
\date{17/04/2024 13:15}

\begin{document}
\maketitle

% Bold title in callout boxes
% But we must be careful: if there are no callout boxes in the document,
% then package tcolorbox has NOT been loaded, and we must refrain from
% setting this up; hence the ifpackageloaded
\makeatletter
\@ifpackageloaded{tcolorbox}
{\tcbset{fonttitle=\bfseries}}
{}
\makeatother


\renewcommand*\contentsname{Índice}
{
\hypersetup{linkcolor=}
\setcounter{tocdepth}{2}
\tableofcontents
}
\bookmarksetup{startatroot}

\chapter*{Apresentação}\label{apresentauxe7uxe3o}
\addcontentsline{toc}{chapter}{Apresentação}

\markboth{Apresentação}{Apresentação}

Este livro eletrônico --- um trabalho em construção --- reúne soluções
dos exercícios do livro \emph{Elementos de Matemática Discreta para
Computação}, de autoria de Anamaria Gomide e Jorge Stolfi (Anamaria
Gomide 2023), disponível em
\url{https://www.ic.unicamp.br/~stolfi/fmc-book/2022-08-24-js/livro.pdf}.

Além das soluções teóricas, incluímos código em SETLX (\emph{Set
Language Extended}), uma linguagem de programação de alto nível
projetada especificamente para resolver problemas envolvendo conjuntos,
relações, funções e outros objetos estudados em um curso de Matemática
Discreta.

Para saber mais sobre SETLX --- incluindo como instalar a liguagem nos
principais sistemas operacionais --- visite
\url{https://randoom.org/Software/SetlX/}. Um tutorial completo de SETLX
está disponível em
\url{https://download.randoom.org/setlX/tutorial.pdf}.

\section*{Agradecimentos}\label{agradecimentos}
\addcontentsline{toc}{section}{Agradecimentos}

\markright{Agradecimentos}

Muitas das soluções foram desenvolvidas ao longo do ano de 2024 pelos
alunos da disciplina de Matemática Discreta do curso de Ciência da
Computação do Pólo de Rio das Ostras da Universidade Federal Fluminense.

\bookmarksetup{startatroot}

\chapter*{2 Teoria dos Conjuntos}\label{teoria-dos-conjuntos}
\addcontentsline{toc}{chapter}{2 Teoria dos Conjuntos}

\markboth{2 Teoria dos Conjuntos}{2 Teoria dos Conjuntos}

\section*{2.1 Conjuntos, elementos e
pertinência}\label{conjuntos-elementos-e-pertinuxeancia}
\addcontentsline{toc}{section}{2.1 Conjuntos, elementos e pertinência}

\markright{2.1 Conjuntos, elementos e pertinência}

\subsection*{Exercício 2.1}\label{exr-2-1}
\addcontentsline{toc}{subsection}{Exercício 2.1}

Escreva os elementos dos conjuntos abaixo:

\begin{enumerate}
\def\labelenumi{\alph{enumi})}
\item
  $\{ x : x \text{ é raiz do polinômio } x^4 - 5x^2 + 6 \}$
\item
  $\{ x^2 + 1 : x \text{ é raiz do polinômio } x^4 - 5x^2 + 6 \}$
\item
  $\{ x \in \{1, 2, 3, 4\} : x \text{ é raiz do polinômio } x^4 - 5x^2 + 6 \}$
\end{enumerate}

\begin{tcolorbox}[enhanced jigsaw, left=2mm, titlerule=0mm, colbacktitle=quarto-callout-important-color!10!white, title={Resposta (a)}, arc=.35mm, opacityback=0, bottomrule=.15mm, breakable, toprule=.15mm, colback=white, coltitle=black, bottomtitle=1mm, rightrule=.15mm, toptitle=1mm, leftrule=.75mm, colframe=quarto-callout-important-color-frame, opacitybacktitle=0.6]

Precisamos usar algum método para resolver a equação

\[
x^4 - 5x^2 + 6 = 0
\]

Uma maneira: se fizermos $y = x^2$, a equação fica

\[
y^2 - 5y + 6 = 0
\]

que tem raízes $y = 2$ e $y = 3$.

Daí, resolvendo $2 = x^2$, temos $x = \pm\sqrt2$.

E resolvendo $3 = x^2$, temos $x = \pm\sqrt3$.

Escrevendo o conjunto como uma enumeração dos elementos:

\[
\left\{ -\sqrt3, -\sqrt2, \sqrt2, \sqrt3 \right\}
\]

\end{tcolorbox}

\begin{tcolorbox}[enhanced jigsaw, left=2mm, titlerule=0mm, colbacktitle=quarto-callout-important-color!10!white, title={Resposta (b)}, arc=.35mm, opacityback=0, bottomrule=.15mm, breakable, toprule=.15mm, colback=white, coltitle=black, bottomtitle=1mm, rightrule=.15mm, toptitle=1mm, leftrule=.75mm, colframe=quarto-callout-important-color-frame, opacitybacktitle=0.6]

Preste atenção: agora, não queremos as raízes, mas sim os valores de
$x^2 + 1$, onde $x$ assume os valores das raízes.

O conjunto poderia ser escrito como

\[
\left\{ x^2 + 1 : x \in \{ -\sqrt3, -\sqrt2, \sqrt2, \sqrt3 \} \right\}
\]

Calculando os valores de $x^2 + 1$, temos:

\begin{longtable*}{rr}
\toprule
\(x\) & \(x^2 + 1\) \\ 
\midrule\addlinespace[2.5pt]
\(-\sqrt{3}\) & \(4\) \\ 
\(-\sqrt{2}\) & \(3\) \\ 
\(\sqrt{2}\) & \(3\) \\ 
\(\sqrt{3}\) & \(4\) \\ 
\bottomrule
\end{longtable*}

Na tabela acima, há elementos repetidos, mas isto não pode acontecer em
um conjunto. Então, a resposta é

\[
\left\{ 3, 4 \right\}
\]

\subsubsection*{Em SETLX}\label{em-setlx}
\addcontentsline{toc}{subsubsection}{Em SETLX}

\begin{Shaded}
\begin{Highlighting}[]
\NormalTok{A := \{ {-}}\KeywordTok{sqrt}\NormalTok{(}\DecValTok{3}\NormalTok{), {-}}\KeywordTok{sqrt}\NormalTok{(}\DecValTok{2}\NormalTok{), }\KeywordTok{sqrt}\NormalTok{(}\DecValTok{3}\NormalTok{), }\KeywordTok{sqrt}\NormalTok{(}\DecValTok{2}\NormalTok{) \};}
\NormalTok{B := \{ x**}\DecValTok{2}\NormalTok{ + }\DecValTok{1}\NormalTok{ : x }\KeywordTok{in}\NormalTok{ A \};}
\KeywordTok{print}\NormalTok{(}\StringTok{"B = "}\NormalTok{, B);}

\CommentTok{// Como SETLX usou ponto flutuante, houve erro.}
\CommentTok{// Vamos arredondar:}
\KeywordTok{print}\NormalTok{( }\StringTok{"B = "}\NormalTok{, \{ }\KeywordTok{round}\NormalTok{(x) : x }\KeywordTok{in}\NormalTok{ B \} );}
\end{Highlighting}
\end{Shaded}

\begin{verbatim}
B = {3.0000000000000004, 3.9999999999999996}
B = {3, 4}
\end{verbatim}

\end{tcolorbox}

\begin{tcolorbox}[enhanced jigsaw, left=2mm, titlerule=0mm, colbacktitle=quarto-callout-important-color!10!white, title={Resposta (c)}, arc=.35mm, opacityback=0, bottomrule=.15mm, breakable, toprule=.15mm, colback=white, coltitle=black, bottomtitle=1mm, rightrule=.15mm, toptitle=1mm, leftrule=.75mm, colframe=quarto-callout-important-color-frame, opacitybacktitle=0.6]

Este item se parece com o item (a), mas há uma diferença importante: os
valores de $x$ --- isto é, os elementos do conjunto --- precisam
pertencer a $\{ 1, 2, 3, 4 \}$. Além disso, os valores de $x$ precisam
ser raízes do polinômio dado.

No item (a), vimos que as raízes são
$-\sqrt3, -\sqrt2, \sqrt2, \text{ e } \sqrt3$. Nenhuma delas pertence ao
conjunto $\{ 1, 2, 3, 4 \}$.

Conclusão: o conjunto do item (c) é vazio.

\subsubsection*{Em SETLX}\label{em-setlx-1}
\addcontentsline{toc}{subsubsection}{Em SETLX}

\begin{Shaded}
\begin{Highlighting}[]
\NormalTok{C := \{ x : x }\KeywordTok{in}\NormalTok{ \{}\DecValTok{1}\NormalTok{, }\DecValTok{2}\NormalTok{, }\DecValTok{3}\NormalTok{, }\DecValTok{4}\NormalTok{\} | x**}\DecValTok{4}\NormalTok{ {-} }\DecValTok{5}\NormalTok{*x**}\DecValTok{2}\NormalTok{ + }\DecValTok{6}\NormalTok{ == }\DecValTok{0}\NormalTok{ \};}
\KeywordTok{print}\NormalTok{(}\StringTok{"C = "}\NormalTok{, C);}
\end{Highlighting}
\end{Shaded}

\begin{verbatim}
C = {}
\end{verbatim}

Observe que a {\hl{notação de SETLX}} divide a especificação do conjunto
{\hl{em três partes}}:

\begin{enumerate}
\def\labelenumi{\arabic{enumi}.}
\tightlist
\item
  A forma geral do elemento: \texttt{x};
\item
  O domínio de onde vêm os valores da variável:
  \texttt{x\ in\ \{1,\ 2,\ 3,\ 4\}};
\item
  A condição que deve ser satisfeita pelos elementos do conjunto:
  \texttt{x**4\ -\ 5*x**2\ +\ 6\ ==\ 0}.
\end{enumerate}

A {\hl{notação do livro}} divide a especificação do conjunto {\hl{em
duas partes}}:

\begin{enumerate}
\def\labelenumi{\arabic{enumi}.}
\tightlist
\item
  A forma geral do elemento e o universo: $x \in \{1, 2, 3, 4\}$;
\item
  A condição que deve ser satisfeita pelos elementos do conjunto: $x$
  \emph{é raiz do polinômio} $x^4 - 5x^2 + 6$.
\end{enumerate}

Lembre-se disso para poder implementar corretamente em SETLX os exemplos
do livro.

\end{tcolorbox}

\subsection*{Exercício 2.2}\label{exr-2-2}
\addcontentsline{toc}{subsection}{Exercício 2.2}

Escreva explicitamente os elementos dos seguintes conjuntos:

\begin{enumerate}
\def\labelenumi{\alph{enumi})}
\item
  $\{ x \in \mathbb{Z}: x^2 - 2x + 1 \leq 0 \}$
\item
  $\{ x \in \mathbb{Z}: 2 \leq x \leq 20 \text{ e } x \text{ é primo} \}$
\item
  $\{ x \in \mathbb{R}: x^2 - 2x = 0 \}$
\end{enumerate}

\begin{tcolorbox}[enhanced jigsaw, left=2mm, titlerule=0mm, colbacktitle=quarto-callout-important-color!10!white, title={Resposta (a)}, arc=.35mm, opacityback=0, bottomrule=.15mm, breakable, toprule=.15mm, colback=white, coltitle=black, bottomtitle=1mm, rightrule=.15mm, toptitle=1mm, leftrule=.75mm, colframe=quarto-callout-important-color-frame, opacitybacktitle=0.6]

Usando seus conhecimentos de Geometria Analítica, você pode traçar o
gráfico da função $f : \mathbb{R}\to \mathbb{R}$ tal que

\[
f(x) = x^2 - 2x + 1
\]

\includegraphics{images/ex-02.02-funcao.png}

O único valor de $x$ para o qual $x^2 - 2x + 1 \leq 0$ é $1$ (que também
é a única raiz desta função).

O conjunto deste item tem os inteiros ($\mathbb{Z}$) como universo, e
$1$ é inteiro. Então, o conjunto é $\{ 1 \}$.

\end{tcolorbox}

\begin{tcolorbox}[enhanced jigsaw, left=2mm, titlerule=0mm, colbacktitle=quarto-callout-important-color!10!white, title={Resposta (b)}, arc=.35mm, opacityback=0, bottomrule=.15mm, breakable, toprule=.15mm, colback=white, coltitle=black, bottomtitle=1mm, rightrule=.15mm, toptitle=1mm, leftrule=.75mm, colframe=quarto-callout-important-color-frame, opacitybacktitle=0.6]

Este é o conjunto dos primos entre $2$ e $20$, inclusive:

\[
\{ 2, 3, 5, 7, 11, 13, 17, 19 \}
\]

\subsubsection*{Em SETLX}\label{em-setlx-2}
\addcontentsline{toc}{subsubsection}{Em SETLX}

\begin{Shaded}
\begin{Highlighting}[]
\CommentTok{// Função para testar se x é primo:}
\NormalTok{primo := }\KeywordTok{procedure}\NormalTok{(x) \{}
  
  \CommentTok{// Para cada inteiro i entre 2 e teto (ceiling) de √x:}
  \KeywordTok{for}\NormalTok{ (i }\KeywordTok{in}\NormalTok{ \{}\DecValTok{2}\NormalTok{..}\KeywordTok{ceil}\NormalTok{(}\KeywordTok{sqrt}\NormalTok{(x))\}) \{}
    
    \CommentTok{// Se o resto de x dividido por i for zero, x não é primo:}
    \KeywordTok{if}\NormalTok{ (x \% i == }\DecValTok{0}\NormalTok{) \{}
      \KeywordTok{return} \KeywordTok{false}\NormalTok{;}
\NormalTok{    \}}
    
\NormalTok{  \}}
  
  \CommentTok{// Se testou todos os valores de i sem dar resto zero, x é primo:}
  \KeywordTok{return} \KeywordTok{true}\NormalTok{;}
  
\NormalTok{\};}

\NormalTok{B := \{ x : x }\KeywordTok{in}\NormalTok{ \{}\DecValTok{2}\NormalTok{..}\DecValTok{20}\NormalTok{\} | primo(x) \};}
\KeywordTok{print}\NormalTok{(}\StringTok{"B = "}\NormalTok{, B);}
\end{Highlighting}
\end{Shaded}

\begin{verbatim}
B = {3, 5, 7, 11, 13, 17, 19}
\end{verbatim}

Você verá, em outros exercícios, maneiras mais curtas para calcular os
primos em um dado intervalo em SETLX. A função acima é a mais parecida
com o que você vai aprender na sua disciplina de programação.

\end{tcolorbox}

\begin{tcolorbox}[enhanced jigsaw, left=2mm, titlerule=0mm, colbacktitle=quarto-callout-important-color!10!white, title={Resposta (c)}, arc=.35mm, opacityback=0, bottomrule=.15mm, breakable, toprule=.15mm, colback=white, coltitle=black, bottomtitle=1mm, rightrule=.15mm, toptitle=1mm, leftrule=.75mm, colframe=quarto-callout-important-color-frame, opacitybacktitle=0.6]

Basta achar as soluções da equação:

\[
x^2 - 2x = 0 \iff x(x - 2) = 0 \iff x = 0 \text{ ou } x = 2
\]

Então, o conjunto é $\{ 0, 2 \}$.

\end{tcolorbox}

\subsection*{Exercício 2.3}\label{exr-2-3}
\addcontentsline{toc}{subsection}{Exercício 2.3}

Determine a cardinalidade dos seguintes conjuntos:

\begin{enumerate}
\def\labelenumi{\alph{enumi})}
\item
  $\{ x \in \mathbb{Z}: -2 \leq x \leq 4 \}$
\item
  $\{ x \in \mathbb{Z}: 10 \leq x^2 \leq 100 \}$
\item
  $\{ x \in \mathbb{R}: x^4 - 5x^2 + 6 = 0 \}$
\item
  $\{ \sin(k\pi/7) : k \in \mathbb{Z}\}$
\end{enumerate}

\begin{tcolorbox}[enhanced jigsaw, left=2mm, titlerule=0mm, colbacktitle=quarto-callout-important-color!10!white, title={Resposta (a)}, arc=.35mm, opacityback=0, bottomrule=.15mm, breakable, toprule=.15mm, colback=white, coltitle=black, bottomtitle=1mm, rightrule=.15mm, toptitle=1mm, leftrule=.75mm, colframe=quarto-callout-important-color-frame, opacitybacktitle=0.6]

Este é o conjunto $A = \{ -2, -1, 0, 1, 2, 3, 4 \}$, que tem $7$
elementos. Logo, $|A| = 7$.

\end{tcolorbox}

\begin{tcolorbox}[enhanced jigsaw, left=2mm, titlerule=0mm, colbacktitle=quarto-callout-important-color!10!white, title={Resposta (b)}, arc=.35mm, opacityback=0, bottomrule=.15mm, breakable, toprule=.15mm, colback=white, coltitle=black, bottomtitle=1mm, rightrule=.15mm, toptitle=1mm, leftrule=.75mm, colframe=quarto-callout-important-color-frame, opacitybacktitle=0.6]

Este é o conjunto dos inteiros cujo quadrado está entre $10$ e $100$,
inclusive.

\textbf{Preste atenção:} números negativos, quando elevados ao quadrado,
resultam em números positivos.

Este conjunto é

\[
B = \{ -10, -9, -8, -7, -6, -5, -4, 4, 5, 6, 7, 8, 9, 10 \}
\]

e $|B| = 14$.

\subsubsection*{Em SETLX}\label{em-setlx-3}
\addcontentsline{toc}{subsubsection}{Em SETLX}

E se os limites não fossem $10$ e $100$?

Vamos fazer uma função mais geral, que recebe dois valores
(\texttt{minimo} e \texttt{maximo}) e retorna o conjunto

\[
\left\{ 
x \in \mathbb{Z}: 
x \in \{ -\sqrt{\texttt{maximo}}, \ldots, +\sqrt{\texttt{maximo}} \}
\text{ e } \texttt{minimo} \leq x^2 \leq \texttt{maximo}
\right\}
\]

Leia com atenção a definição acima. Por que o domínio é
$\{ -\sqrt{\texttt{maximo}}, \ldots, +\sqrt{\texttt{maximo}} \}$?

\begin{Shaded}
\begin{Highlighting}[]
\NormalTok{calcular := }\KeywordTok{procedure}\NormalTok{(minimo, maximo) \{}
  
\NormalTok{  extremo\_esquerdo := {-}}\KeywordTok{floor}\NormalTok{(}\KeywordTok{sqrt}\NormalTok{(maximo));}
\NormalTok{  extremo\_direito  :=  }\KeywordTok{floor}\NormalTok{(}\KeywordTok{sqrt}\NormalTok{(maximo));}
  
  \KeywordTok{return}\NormalTok{ \{ }
\NormalTok{    x : }
\NormalTok{      x }\KeywordTok{in}\NormalTok{ \{ extremo\_esquerdo..extremo\_direito \} |}
\NormalTok{        minimo \textless{}= x**}\DecValTok{2}\NormalTok{ \&\& x**}\DecValTok{2}\NormalTok{ \textless{}= maximo}
\NormalTok{  \};}
  
\NormalTok{\};}

\CommentTok{// Vários conjuntos, com valores diferentes:}
\KeywordTok{print}\NormalTok{(}\StringTok{"B  = "}\NormalTok{, calcular(}\DecValTok{10}\NormalTok{, }\DecValTok{100}\NormalTok{));}
\KeywordTok{print}\NormalTok{(}\StringTok{"B2 = "}\NormalTok{, calcular(}\DecValTok{5}\NormalTok{, }\DecValTok{50}\NormalTok{));}
\KeywordTok{print}\NormalTok{(}\StringTok{"B3 = "}\NormalTok{, calcular(}\DecValTok{0}\NormalTok{, }\DecValTok{5}\NormalTok{));}
\KeywordTok{print}\NormalTok{(}\StringTok{"B4 = "}\NormalTok{, calcular(}\DecValTok{0}\NormalTok{, }\DecValTok{0}\NormalTok{));}

\CommentTok{// B5 está correto?}
\KeywordTok{print}\NormalTok{(}\StringTok{"B5 = "}\NormalTok{, calcular(}\DecValTok{10}\NormalTok{, }\DecValTok{0}\NormalTok{));}

\KeywordTok{print}\NormalTok{(}\StringTok{"\textbackslash{}n"}\NormalTok{);}

\CommentTok{// Nossa função não é muito robusta. }
\CommentTok{// Ela quebra se maximo for negativo. Por quê?}
\KeywordTok{print}\NormalTok{(}\StringTok{"B6 = "}\NormalTok{, calcular({-}}\DecValTok{20}\NormalTok{, {-}}\DecValTok{10}\NormalTok{));}
\end{Highlighting}
\end{Shaded}

\begin{verbatim}
B  = {-10, -9, -8, -7, -6, -5, -4, 4, 5, 6, 7, 8, 9, 10}
B2 = {-7, -6, -5, -4, -3, 3, 4, 5, 6, 7}
B3 = {-2, -1, 0, 1, 2}
B4 = {0}
B5 = {}


Error in "print("B6 = ", calcular(-20, -10))":
Error in "calcular(-20, -10)":
Error in "extremo_esquerdo := -floor(sqrt(maximo))":
Error in "-floor(sqrt(maximo))":
Error in "floor(sqrt(maximo))":
Error in "sqrt(maximo)":
Result of this operation is undefined/not a number.

Replay: 
2.4: sqrt(maximo) FAILED 
2.3: maximo <~> -10
2.2: sqrt <~> procedure(x) { /* predefined procedure `sqrt' */ }
2.1: floor <~> procedure(numberValue) { /* predefined procedure `floor' */ }
1.8: calcular(-20, -10) FAILED 
1.7: -10 <~> -10
1.6: 10 <~> 10
1.5: -20 <~> -20
1.4: 20 <~> 20
1.3: calcular <~> procedure(minimo, maximo) { extremo_esquerdo := -floor(sqrt(maximo)); extremo_direito := floor(sqrt(maximo)); return {x : x in {extremo_esquerdo .. extremo_direito} | minimo <= x ** 2 && x ** 2 <= maximo}; }
1.2: "B6 = " <~> "B6 = "
1.1: print <~> procedure(*value) { /* predefined procedure `print' */ }
\end{verbatim}

\end{tcolorbox}

\begin{tcolorbox}[enhanced jigsaw, left=2mm, titlerule=0mm, colbacktitle=quarto-callout-important-color!10!white, title={Resposta (c)}, arc=.35mm, opacityback=0, bottomrule=.15mm, breakable, toprule=.15mm, colback=white, coltitle=black, bottomtitle=1mm, rightrule=.15mm, toptitle=1mm, leftrule=.75mm, colframe=quarto-callout-important-color-frame, opacitybacktitle=0.6]

Este é o conjunto $A$ do \hyperref[exr-2-1]{Exercício 2.1 (a)}, que tem
$4$ elementos. logo, $|A| = 4$.

\end{tcolorbox}

\bookmarksetup{startatroot}

\chapter*{Referências}\label{referuxeancias}
\addcontentsline{toc}{chapter}{Referências}

\markboth{Referências}{Referências}

\phantomsection\label{refs}
\begin{CSLReferences}{1}{0}
\bibitem[\citeproctext]{ref-anamaria23:_elemen_matem_discr_comput}
Anamaria Gomide, Jorge Stolfi. 2023. \emph{Elementos de Matemática
Discreta para Computação}.
\url{https://www.ic.unicamp.br/~stolfi/fmc-book/2022-08-24-js/livro.pdf}.

\end{CSLReferences}



\end{document}
